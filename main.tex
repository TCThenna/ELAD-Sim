\documentclass[preprint]{elsarticle}
%\documentclass[review,number,sort&compress]{elsarticle}

\usepackage{amssymb}
\usepackage{amsmath}
\usepackage{graphicx}
\usepackage[load-configurations=abbreviations]{siunitx}
\usepackage[european]{circuitikz}
\usepackage{xstring}
\usepackage{tikz}
\usepackage{verbatim}
\usepackage[active]{srcltx}
\usepackage{multicol}
%\usepackage{lipsum}
\usepackage{graphicx}
\usepackage{nicefrac}
\usepackage{fancyhdr}
\usepackage[english]{babel}
\usepackage{color}
%% The amsthm package provides extended theorem environments
%% \usepackage{amsthm}
\newcommand{\sigmabin}{\ensuremath{\mathnormal{\sigma_{\textrm{bin}}}}}
\newcommand{\sigmaimp}{\ensuremath{\mathnormal{\sigma_{\textrm{imp}}}}}
\newcommand{\deltaimp}{\ensuremath{\mathnormal{\delta_{\textrm{imp}}}}}

\newcommand{\vect}[1]{\boldsymbol{#1}}

%% The lineno packages adds line numbers. Start line numbering with
%% \begin{linenumbers}, end it with \end{linenumbers}. Or switch it on
%% for the whole article with \linenumbers after \end{frontmatter}.
\usepackage{lineno}


%linenumbers
\setpagewiselinenumbers
\modulolinenumbers[5]

\begin{document}

\begin{frontmatter}


\title{Simulation of enhanced lateral drift sensors}

\author[desy]{A.~Velyka}
\ead{a.v@desy.de}

\author[cern]{S.~Spannagel\corref{cor2}}
\ead{s.s@cern.ch}

\address[cern]{Route de , Geneva, Switzerland}

\author[desy]{H.~Jansen\corref{cor1}}
\ead{hendrik.jansen@desy.de}

\cortext[cor1]{Corresponding author}
\address[desy]{Notkestr. 85, 22607 Hamburg, Germany}

\begin{abstract}
A novel concept of a layer-wise produced semiconductor sensor for precise particle tracking is proposed herein. 
In contrast to common semiconductor sensors, local regions with increased doping concentration deep in the bulk termed \textit {charge guides}
 increase the lateral drift of free charges on their way to the read-out electrode. 
This lateral drift enables charge sharing independent of the incident position of the traversing particle. 
With a regular grid of charge guides the lateral charge distribution resembles a normalised Pascal's triangle for particles that are stopped in depths lower than the depth of the first layer of the charge guides. 
%For different configurations and patterns, a combination of binomial distributions is thought to describe the lateral charge distribution. 
For minimum ionising particles a sum of binomial distributions describes the lateral charge distribution. 
This concept decouples the achievable sensor resolution from the pitch size as the characteristic length is replaced by the lateral distance of the charge guides. 
\end{abstract}

\begin{keyword}
Semiconductor detector \sep deep bulk engineering \sep sensor concept \sep precision particle tracking
\PACS 29.40.Wk \sep 72.20.Fr
\end{keyword}

\end{frontmatter}

\linenumbers
\section{Introduction}
The necessity for charged particle detectors with improved spatial resolution drives many silicon-detector R\&D projects in various areas of research towards small pitch sizes.
However, this evolutionary approach increases the number of read-out channels resulting in various disadvantages. 
In this work, a new sensor concept exemplifying a strategy for high-precision silicon sensors is proposed that avoids the small pitch paradigm and decouples the sensor resolution from its pitch size and thickness.
Implants deep inside the bulk of this new sensor are foreseen to laterally enlarge the charge cloud produced by a traversing particle, thus enhancing charge sharing between neighbouring strips/pixels. 
With appropriately placed deep implants, the lateral drift can be optimised towards optimal charge sharing and hence resolution of the incident position. 
In this concept of an enhanced lateral drift (ELAD) sensor, the linearity of the charge sharing and Landau fluctuations dominate the sensor resolution rather than the pitch size
 – even for thin sensors, where usually almost no charge sharing takes place.

 ~\cite{Moliere:1948zz,PhysRev.89.1256}

\section{The concept of pLAD sensors}


\begin{figure}[t]
  \centering
  \includegraphics[width=0.99\textwidth]{figures/2Dxy} %\put(-35,0){$[\si{\um}]$} \put(-335,180){$[\si{\um}]$}
  \caption{The electric field distribution in a pLAD sensor.
  The black rectangles in the centre of the figure mark the deep p$^+$-implants (charge guides) in the p-bulk.
  Units are in micrometer.}
\label{fig:potential}
\end{figure}

\section{Summary}
A novel concept of semiconductor particle sensors was proposed that decouples the achievable spatial resolution of the sensor from the pitch size of the read-out electronics. 
For this, implants acting as charge guides deep in the bulk of the sensor need to be engineered appropriately for enhanced lateral charge drift. 
The characteristic length affecting the spatial resolution is the distance between the charge guides rather than the pitch size. 
Preliminary simulations show the feasibility of enabling lateral charge drift by local high dose implants of Boron atoms in a p-type sensors. 

\section*{Acknowledgement}
This project is funded via PIER Hamburg, grant no. PIF-2015-37.


\bibliographystyle{unsrt}
\bibliography{bibtex/refs}


\end{document}
