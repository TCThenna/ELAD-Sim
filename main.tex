\documentclass[preprint]{elsarticle}
%\documentclass[review,number,sort&compress]{elsarticle}

\usepackage{amssymb}
\usepackage{amsmath}
\usepackage{graphicx}
\usepackage[load-configurations=abbreviations]{siunitx}
\usepackage[european]{circuitikz}
\usepackage{xstring}
\usepackage{tikz}
\usepackage{verbatim}
\usepackage[active]{srcltx}
\usepackage{multicol}
%\usepackage{lipsum}
\usepackage{graphicx}
\usepackage{nicefrac}
\usepackage{fancyhdr}
\usepackage[english]{babel}
\usepackage{color}
%% The amsthm package provides extended theorem environments
%% \usepackage{amsthm}
\newcommand{\sigmabin}{\ensuremath{\mathnormal{\sigma_{\textrm{bin}}}}}
\newcommand{\sigmaimp}{\ensuremath{\mathnormal{\sigma_{\textrm{imp}}}}}
\newcommand{\deltaimp}{\ensuremath{\mathnormal{\delta_{\textrm{imp}}}}}

\newcommand{\vect}[1]{\boldsymbol{#1}}

%% The lineno packages adds line numbers. Start line numbering with
%% \begin{linenumbers}, end it with \end{linenumbers}. Or switch it on
%% for the whole article with \linenumbers after \end{frontmatter}.
\usepackage{lineno}


%linenumbers
\setpagewiselinenumbers
\modulolinenumbers[5]

\begin{document}

\begin{frontmatter}


\title{Simulation of enhanced lateral drift sensors}

\author[desy]{A.~Velyka}
\ead{anastasiia.velyka@desy.de}
\address[desy]{Notkestr. 85, 22607 Hamburg, Germany}

\author[cern]{S.~Spannagel\corref{cor2}}
\ead{simon.spannagel@cern.ch}

\address[cern]{Route de , Geneva, Switzerland}

\author[desy]{H.~Jansen\corref{cor1}}
\ead{hendrik.jansen@desy.de}

\cortext[cor1]{Corresponding author}


\begin{abstract}
A novel concept of a layer-wise produced semiconductor sensor for precise particle tracking is proposed herein. 

\end{abstract}

\begin{keyword}
Semiconductor detector \sep deep bulk engineering \sep conceptual design \sep precision particle tracking
\PACS 29.40.Wk \sep 72.20.Fr
\end{keyword}

\end{frontmatter}

\linenumbers
\section{Introduction}
Precision measurements of Higgs decays at a future collider necessitate charged particle detectors with single point resolutions of about $\SI{3}{\um}$ at sensor thickness at or below $\SI{100}{\um}$.~\cite{DominiksCLICNote}
These requirements drive the mainstream  of detector R\&D for collider experiments towards smaller and smaller sensor unit-cells, or pitches.
However, this evolutionary approach increases the number of read-out channels resulting in various disadvantages. 
In this work, a new sensor concept is presented that exemplifies a strategy towards achieving the theoretical optimum of position resolution. 
It avoids the small pitch paradigm as much as possible and renders optimal spatial resolution possible, even at sensor thickness of $\SI{50}{\um}$.
In this new approach, implants (volumes with higher doping concentration compared to the background concentration) deep inside the bulk guide the charge cloud produced by a traversing particle laterally towards
 the boarder between two unit-cells, thus enhancing charge sharing between neighbouring strips/pixels. 
With appropriately placed deep implants, the lateral drift can be optimised towards optimal, i.e.\ linear charge sharing and hence optimal resolution of the incident position at a given pitch size. 
In this concept of an enhanced lateral drift (ELAD) sensor, the degree of linearity of the charge sharing and Landau fluctuations limit the sensor resolution. %FIXME check if this is true :D
 
This is some reference that can be deleted~\cite{Moliere:1948zz,PhysRev.89.1256}

\section{The concept of ELAD sensors}

conceptual discussion, position resolution, drift and diffusion, deep implants, bulk engineering, n- and p-ELAD, ...

\section{Device optimisation of an ELAD sensor}

Initial geometry, optimisation strategy, static and transient, ...

\section{Monte Carlo studies}

AP2, which modules, manipulations of the TCAD field, MIPs, drift and diff, 

\section{Results}

tcad eta functions, AP2 eta functions

\section{Discussion}

single point res a.f.o various parameters: concentration, bias voltage, thickness, maybe turn on/off Landau if possible

\begin{figure}[t]
  \centering
  \includegraphics[width=0.8\textwidth]{figures/dummy} %\put(-35,0){$[\si{\um}]$} \put(-335,180){$[\si{\um}]$}
  \caption{dummy}
\label{fig:dummy}
\end{figure}

\section{Conclusion}


\section*{Acknowledgement}
This project is funded via PIER Hamburg, grant no. PIF-2015-37.

\small
\section*{References}
\bibliographystyle{unsrt}
\bibliography{bibtex/refs}


\end{document}
